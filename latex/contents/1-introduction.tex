\introduction % Структурный элемент: ВВЕДЕНИЕ
Постоянный и впечатляющий прогресс в области компьютерного оборудования за последние годы привёл к широкому распространению мощных и доступных компьютеров, средств сбора данных и устройств хранения информации. Благодаря этому прогрессу индустрия баз данных и информационных технологий получила мощный импульс для создания огромного количества баз данных и информационных хранилищ, которые используются для управления транзакциями, поиска информации и анализа данных.

Таким образом, технологическое развитие обеспечило стремительный рост объёма текстовых документов, доступных в интернете, цифровых библиотеках и репозиториях, новостных источниках, корпоративных интрасетях и оцифрованной личной информации, такой как блоги и электронные письма. С увеличением количества электронных документов становится всё сложнее вручную их организовывать, анализировать и представлять эффективно. Это породило вызовы в сфере эффективной автоматической организации текстовых документов.

Решение данной проблемы связано с применением передовых технологий, таких как машинное обучение, искусственный интеллект и методы кластерного анализа. Эти методы позволяют автоматизировать процессы организации информации, повышая скорость обработки данных, точность результатов и удобство их последующего использования.





% ------
% С быстрым развитием цифровых технологий доступ к информации стал более распространённым аспектом повседневной жизни, чем когда-либо прежде. Бизнес и частные лица теперь имеют доступ к огромному объёму данных — миллиардам документов, видео- и аудиофайлов в интернете. Большинство компаний функционируют в тесно взаимосвязанной среде данных, связанной с множеством информационных источников, что позволяет получать доступ к значительным объёмам информации. Однако, несмотря на быстрый рост вычислительных возможностей, сам по себе огромный объём данных часто затрудняет преобразование разрозненной информации в полезные сведения, практические знания и конкретные действия. Одним из устоявшихся решений является использование машинного обучения, в частности методов кластеризации.

% Алгоритмы кластеризации — это алгоритмы машинного обучения, которые стремятся группировать схожие данные на основе определённых критериев, выявляя тем самым естественные структуры или закономерности в наборе данных. Основная цель — разделить данные на подгруппы (кластеры), элементы которых внутри одного кластера более похожи друг на друга, чем на элементы других кластеров. Методы кластеризации способствуют развитию различных областей, таких как поиск информации, рекомендательные системы и выявление тем.

% Алгоритмы кластеризации широко применяются в повседневной жизни. Их используют для фильтрации спама, в рекомендательных системах, для сегментации клиентов в целях таргетированного маркетинга, в обработке изображений — для группировки изображений по визуальному сходству и многого другого. Алгоритмы кластеризации применимы как к текстовым данным, так и к аудио-, видео- и графическим материалам. Аудиокластеризация использует акустические характеристики для группировки файлов и может применяться, например, для определения музыкальных жанров. Кластеризация видео помогает в создании рекомендаций и аннотаций, группируя контент по визуальным или тематическим признакам. В анализе изображений кластеризация важна для сегментации и поиска по содержанию. Алгоритмы кластеризации также используются для выявления аномалий — будь то в сетевом трафике, финансовых транзакциях или медицинских записях. Их универсальность подчёркивает значимость этих алгоритмов в выявлении закономерностей из самых разных типов данных.
% Два основных типа алгоритмов кластеризации, исходя из наличия размеченных данных и способа обучения, следующие:
% \begin{itemize}[label=., leftmargin=3em]
%     \item Полу-контролируемая кластеризация (Semi-supervised Clustering):
    
% В этой категории для каждой обучающей выборки предоставляется известная метка кластера. Алгоритм анализирует закономерности в обучающих данных и затем учится присваивать новые точки данных соответствующим кластерам. Примеры таких алгоритмов включают Constrained K-Means [1] и Semi-Supervised Fuzzy C-Means (SSFCM)

%     \item Ненаблюдаемое (без учителя) обучение (Unsupervised Learning):
    
% В этой категории метки классов (кластеров) не предоставляются. Алгоритм самостоятельно выявляет закономерности и структуры в данных, группируя похожие образцы на основе сходства признаков без предварительного знания о принадлежности к группам. Обычно общее количество кластеров в наборе данных заранее неизвестно. Примеры алгоритмов в этой категории включают K-Means [3, 4], DBSCAN (Density-Based Spatial Clustering of Applications with Noise) [5] и Fuzzy C-Means (FCM) [6].
% \end{itemize}


% Для определения подходящего количества кластеров могут использоваться различные методы, такие как метод локтя (elbow method), коэффициент силуэта (silhouette score), статистика разрыва (gap statistics) 
