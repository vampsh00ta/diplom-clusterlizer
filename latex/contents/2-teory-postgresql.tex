\section{База данных}
\subsection{PostgreSQL}

PostgreSQL— это мощная объектно-реляционная система управления базами данных (ORDBMS) с открытым исходным кодом, известная своей надёжностью, сохранностью данных и богатым набором возможностей. Она поддерживает расширенные типы данных, сложные запросы, внешние ключи, триггеры, представления, а также процедурные языки для хранения процедур. PostgreSQL легко расширяется: пользователи могут добавлять новые функции, типы данных и другие возможности. Жёсткое соблюдение стандартов SQL и поддержка свойств ACID (атомарность, согласованность, изолированность, надёжность) делают PostgreSQL идеальным выбором для разработчиков и компаний, которым нужна масштабируемая, эффективная и безопасная база данных.

Выбор PostgreSQL в качестве СУБД даёт уникальное сочетание преимуществ, отвечающих самым разным требованиям к управлению данными. Одно из ключевых достоинств — превосходная поддержка продвинутых типов данных и расширенного функционала. В их числе — нативная работа с JSON, геометрическими данными и пользовательскими типами, что обеспечивает гибкое и эффективное хранение и извлечение информации под самые сложные и разнообразные модели данных. Способность PostgreSQL легко обрабатывать сложные запросы, транзакции и масштабные операции хранилищ данных делает СУБД отличным выбором для приложений, требующих глубокой аналитики, обработки в реальном времени и высокой целостности данных.

PostgreSQL обладает множеством преимуществ, которые делают её превосходным решением для самых разных задач управления данными. СУБД известна своей надёжностью, богатым функционалом и гибкостью, позволяя эффективно решать сложные задачи обработки и управления данными. Система ориентирована на поддержку крупномасштабных приложений и хранилищ, что даёт возможность строить масштабируемые и защищённые решения.

Одной из сильных сторон PostgreSQL является исключительная гибкость и расширяемость: пользователь может адаптировать базу под свои потребности. СУБД поддерживает широкий спектр встроенных и пользовательских типов данных и предоставляет несколько процедурных языков для написания хранимых процедур. Благодаря этому можно расширять возможности системы собственными функциями, операторами и новыми языками, обеспечивая эволюцию PostgreSQL вместе с развитием проекта.

PostgreSQL также отличается высокими производительностью и масштабируемостью, легко справляясь с большими объёмами данных и множеством параллельных транзакций. СУБД применяет продвинутые методы оптимизации для эффективного хранения и извлечения информации, что делает её подходящей для высоконагруженных систем. Архитектура предусматривает горизонтальное масштабирование, партиционирование и репликацию, позволяя приложениям без труда расти вместе с увеличением объёма данных и числа пользователей.

Благодаря мощным средствам оптимизации запросов PostgreSQL эффективно выполняет сложные выборки. Среди её инструментов — index-only scans, bitmap heap scans и генетический планировщик запросов, которые сокращают время их выполнения. Всё это делает PostgreSQL идеальной для задач аналитики и бизнес-интеллекта, где требуются быстрое и точное извлечение данных.

