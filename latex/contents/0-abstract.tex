\abstract % Структурный элемент: РЕФЕРАТ

\keywords{КЛАСТЕРИЗАЦИЯ, СЕМАНТИЧЕСКИЙ АНАЛИЗ, МИКРОСЕРВИСЫ, HDBSCAN, SBERT, PostgreSQL, RABBITMQ, ГРАФОВЫЙ ИНТЕРФЕЙС}

В работе разработана и реализована высокопроизводительная система кластеризации документов, принимающая файлы форматов PDF и DOCX, которая генерирует кластеры из извлечённых текстовых данных и визуализирует результаты в графовом интерфейсе.

Практическая часть включает разработку приложения, построенного на микросервисной архитектуре, которая обеспечивает гибкость, масштабируемость и изоляцию компонентов. Система состоит из следующих компонентов: хранилище данных (S3), брокер сообщений (RabbitMQ), база данных (PostgreSQL), сервисы Public API и Clustulizer, а также пользовательский интерфейс.

Public API реализован на языке Golang с использованием фреймворка Fiber и обеспечивает обработку HTTP-запросов пользователей. Сервис Clustulizer реализован на Python, использует модель SBERT-LARGE-NLU-RU для генерации эмбеддингов и алгоритм кластеризации HDBSCAN.

Оценка качества кластеризации показала высокий уровень с использованием метрик Silhouette Score (0.923) и низкий индекс Davies-Bouldin (0.107), подтверждая эффективность предложенного решения. Среднее время обработки документов составляет 120 мс, с пропускной способностью до 40 документов в секунду.

Пользовательский интерфейс системы обеспечивает наглядное представление результатов в виде интерактивного графа, позволяя пользователям легко ориентироваться в результатах обработки.