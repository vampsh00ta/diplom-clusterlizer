\conclusion

 В рамках данной работы была разработана система тематической кластеризации документов, использующая передовые методы семантического анализа и современные алгоритмы машинного обучения. Проведённые тесты и оценка результатов с помощью специализированных метрик подтвердили высокое качество и производительность предложенного решения. Использование микросервисной архитектуры обеспечило необходимую гибкость и масштабируемость, позволяя легко адаптировать систему под различные требования и нагрузки.

Применение модели SBERT-LARGE-NLU-RU и алгоритма кластеризации HDBSCAN продемонстрировало эффективность и высокую точность в задачах обработки и тематического анализа текстовых данных. Интерфейс, представленный в виде графа, обеспечивает удобную визуализацию и простое взаимодействие пользователей с результатами кластеризации.

В дальнейшем планируется совершенствовать систему за счёт расширения функционала, оптимизации алгоритмов обработки данных, улучшения интерфейса и повышения общей производительности. Дополнительным направлением развития может стать интеграция с другими информационными системами и автоматизация процессов мониторинга и анализа больших объёмов текстовых данных.